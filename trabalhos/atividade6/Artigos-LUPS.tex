\documentclass[12pt,a4paper,compsoc]{IEEEtran}
\usepackage[utf8]{inputenc}
\usepackage[brazilian]{babel}
\usepackage[dvips]{graphicx}
\usepackage{epsfig}
\usepackage{amsfonts, color}
\usepackage{subfigure}
\usepackage{eurosym}
\usepackage[T1]{fontenc} 
\usepackage{ae}
\usepackage{cite}

%------------------------------------------------------------------------------
% Template dos artigos da Revista LUPS
%------------------------------------------------------------------------------

%------------------------------------------------------------------------------

\begin{document}

%------------------------------------------------------------------------------
%%%%%%%%%%%%%%%%%%%%%%%%%%%%%%%%%%%%%%%%%%%%%%%%%%%%%%%%%%%%%%%%%%%%%%%%%%%%%%%
% Estas informaes devem ser alteradas pelo editor da revista
\pubid{LABORATORY OF UBIQUITOUS AND PARALLEL SYSTEMS~\copyright~LUPS} 
\renewcommand{\leftmark}{REVISTA~LUPS,~VOL.~2, NO.~1,~DEZEMBRO~2013}
%%%%%%%%%%%%%%%%%%%%%%%%%%%%%%%%%%%%%%%%%%%%%%%%%%%%%%%%%%%%%%%%%%%%%%%%%%%%%%%
%------------------------------------------------------------------------------

% Ttulo e autores do artigo

\title{Template em LATEX para os artigos da revista LUPS}

\author{Nome Autor 1, UFPel; Nome Autor 2, UFPel

\IEEEcompsocitemizethanks{\IEEEcompsocthanksitem \textbf{Caue Duarte}: Programa de Pós-Graduao em Computação,  Universidade Federal de Pelotas - UFPel, Centro de Desenvolvimento Tecnológigo - CDTec.
% Para uma quebra de linha, deve-se utilizar o \protect antes do \\, caso contrrio vai dar erro.
\protect\\ E-mail: e-mailautor1@inf.ufpel.edu.br}

% No caso de mais autores, basta repetir:
\IEEEcompsocitemizethanks{\IEEEcompsocthanksitem \textbf{Alexandre Gomes da Costa}: Programa de Pós-Graduao em Computação,  Universidade Federal de Pelotas - UFPel, Centro de Desenvolvimento Tecnológigo - CDTec.
\protect\\ E-mail: alexandre.costa@inf.ufpel.edu.br}
}

%------------------------------------------------------------------------------

\IEEEcompsoctitleabstractindextext{%
\begin{abstract}
Resumo no  uma introduo ao artigo, mas sim uma descrição sumria da sua totalidade, na qual se procura realar os aspectos mencionados. Deve ser discursivo, e no apenas uma lista dos tpicos que o artigo cobre. Deve-se entrar na essncia do resumo logo na primeira frase, sem rodeios introdutrios nem recorrendo  frmula estafada "Neste artigo...". Este deve ser escrito em um pargrafo nico e conter entre 120 e 180 palavras. Devem-se evitar em um resumo: (i) Smbolos e contraes que no sejam de uso corrente; (ii) Frmulas, equaes, diagramas etc., que no sejam absolutamente necessrios. No citar referncias bibliogrficas no resumo. Logo aps o resumo devem vir as palavras-chave, que so palavras representativas do contedo do documento. As palavras-chave devem ser separadas entre si por vrgula e finalizadas por ponto. No mnimo 3 e no mximo 5 palavras.
\end{abstract}

\begin{IEEEkeywords}
Palavra-chave 1, Palavra-chave 2, Palavra-chave 3, Palavra-chave 4.
\end{IEEEkeywords}}

\maketitle

%------------------------------------------------------------------------------


\section{Introdução}


\IEEEPARstart{E}{}ste arquivo  apenas uma demonstrao de um artigo da Revista LUPS. Seu objetivo  servir como um "arquivo de partida" para os autores, pois traz uma srie de itens comuns, tais como sees, subsees, tabelas, imagens, entre outros que certamente sero utilizados no decorrer da redao de cada captulo. Sucesso a todos na escrita.


Este arquivo  apenas uma demonstrao de um artigo da Revista LUPS. Seu objetivo  servir como um "arquivo de partida" para os autores, pois traz uma srie de itens comuns, tais como sees, subsees, tabelas, imagens, entre outros que certamente sero utilizados no decorrer da redao de cada captulo. Sucesso a todos na escrita.


Este arquivo  apenas uma demonstrao de um artigo da Revista LUPS. Seu objetivo  servir como um "arquivo de partida" para os autores, pois traz uma srie de itens comuns, tais como sees, subsees, tabelas, imagens, entre outros que certamente sero utilizados no decorrer da redao de cada captulo. Sucesso a todos na escrita.


Este arquivo  apenas uma demonstrao de um artigo da Revista LUPS. Seu objetivo  servir como um "arquivo de partida" para os autores, pois traz uma srie de itens comuns, tais como sees, subsees, tabelas, imagens, entre outros que certamente sero utilizados no decorrer da redao de cada captulo. Sucesso a todos na escrita.


%------------------------------------------------------------------------------

\section{Computação Ubíqua}



O termo Computação Ubíqua foi introduzido pelo cientista Mark Weiser, que no início dos anos 90, já previa um aumento nas funcionalidades e na disponibilidade de serviços de computação para os usuários finais, entretanto com a visibilidade destes serviços sendo a menor possível. Deste modo, a computação no seria exclusividade de um computador, mas de diversos dispositivos conectados entre si.
A Computação Ubíqua, nesta perspectiva, não significa um computador que possa ser transportado para diferentes lugares. Mesmo o mais funcional notebook, com acesso a Internet ainda foca a atenção do usuário num único equipamento. Comparando  escrita, carregar este poderoso notebook  como carregar um livro muito importante. Por mais significativo que seja este livro, no significa capturar o real poder da Literatura (WEISER, 1991).
Uma analogia oportuna seria o surgimento da escrita moderna. Conhecimento que durante muito tempo foi acessível exclusivamente aos mais conceituados especialistas das letras, hoje está integralmente imerso em nosso cotidiano e imperceptivelmente, constituindo uma tecnologia consumida em larga escala.
A concretização do pensamento de Weiser, e consequentemente a ubiquidade da informática, ter atingido sua plenitude quando a computação for aplicada com a mesma naturalidade que hoje  utilizada a língua escrita e os motores, agora elétricos e embarcados, para realização de atividades do cotidiano, ou seja, a comunidade ter alcançado a era da computação ubíqua quando o computador deixar de ser peça única e de uso genérico e multiplicar-se para usos especializados.
O computador uma interface de comunicação e, como todas as interfaces consolidadas, ele tende a ser imperceptível. Por exemplo, os óculos são uma interface entre o ser humano e o mundo mas, o ser humano não concentra suas atenções no óculos, mas no mundo. Uma pessoa cega, ao utilizar sua bengala, percebe o mundo ao seu redor, e no a bengala. Sob esta premissa, o uso das interfaces passa a tornar-se inconsciente (WEISER, 1993).
A Computação Ubíqua entendida como sendo o terceiro grande paradigma computacional, precedido pelo império dos mainframes e pela onda da computação pessoal (WEISER; BROWN, 1997). Com foco em demandas administrativas no passado e hoje presente em cenários específicos, os computadores de grande porte foram concebidos sobre uma arquitetura onde, poucas máquinas de imenso poder de processamento, são compartilhadas simultaneamente por inúmeros usuários. Com o domínio dos microcomputadores, observamos a máquina como peça de uso pessoal e exclusivo. Por fim, na Computação Ubíqua, veremos a tecnologia formando uma malha de dispositivos inteligentes em torno de cada indivíduo.
As aplicações ubíquas, em uma visão mais ampla, devem prever a mobilidade de equipamentos e usuários, denominada mobilidade física, e também dos componentes da aplicação e serviços, chamada de mobilidade lógica. Para isso, as aplicações devem ter o estilo siga-me, facultando que o usuário possa acessar seu ambiente computacional independente da localização, do tempo e do dispositivo utilizado (YAMIN, 2004).
Para tal, as aplicações precisam "entender" e se adaptar ao ambiente, compreendendo o contexto em que esto inseridas (MACIEL; ASSIS, 2004). Essa nova classe de sistemas computacionais, conscientes ao contexto, abre perspectivas para o desenvolvimento de aplicações muito mais ricas, elaboradas e complexas, que exploram a natureza dinâmica e a mobilidade do usuário. Um desafio central na programação deste tipo de aplicação  possibilitar que as mesmas se adaptem continuamente ao ambiente e permaneçam funcionando mesmo quando o indivíduo se movimentar ou trocar de dispositivo (GRIMM; BERSHAD, 2003; COSTA; YAMIN; GEYER, 2008).
A Computação Ubíqua considerando esta perspectiva, constitui um ambiente altamente distribuído, heterogêneo, dinâmico, móvel, mutável e com forte interação entre homem e máquina (AUGUSTIN, 2003). Neste contexto, o usuário desloca-se e comunica-se com os recursos computacionais do ambiente em que se encontra, promovendo adaptação dinâmica dos serviços e aplicações, de tal forma que as suas necessidades sejam satisfeitas, da maneira mais natural e transparente possíveis.


%------------------------------------------------------------------------------

\section{Conscincia de Situação}

%Aqui falar de Conscincia de Situao de modo geral e introdutório e conceitos.
%Survey
%Trabalho Ana
%Trabalho Ricardo






%So o trabalho do ricardo {
\subsection{Passos para aquisião da conciencia de situaao}

\subsubsection{Percepao}
\subsubsection{Compreenao}
\subsubsection{Projeao}
%}

%Survey{
\subsection{Sensores e dados}

\subsection{Tecnicas de Identificaao de Situacao}

\subsubsection{Learning-based approaches}

Movendo-se para o lado direito da figura. 2, os avanos em tecnologias de sensores impulsionar a implantação de uma ampla gama de sensores, que no entanto prejudicaram o desempenho das abordagens baseadas em especificações. Menos visível para apenas usar o conhecimento de especialistas para definir as especificações adequadas de situações a partir de um grande número de dados de sensores ruidosos. Para resolver este problema, técnicas de aprendizado de máquina e mineração de dados são emprestados para explorar as relações de associação entre os dados e situações de sensores. Uma grande quantidade de pesquisas têm sido realizadas na rea de reconhecimento de atividade em ambientes inteligentes recentemente.
Uma série de modelos derivados bayesianos são popularmente aplicados, incluindo Naive Bayes [30,31] e redes Bayesianas [32,22] com a força de codificação causal (dependência) relacionamentos e dinâmicas Bayesian Networks [33], Hidden Markov Models [34, 35] e Campos Aleatórios Condicionais [36,10], com a força de codificação de relações temporais. Inspirado da modelagem da linguagem, abordagens baseadas em gramática como (estocásticos) gramáticas livres de contexto são aplicados em representar a semântica estrutural complexo de processos em situações hierárquicas [37-39]. As árvores de decisão [40,5], redes neurais [41], e Support Vector Machines [42,43] como uma outra filial em técnicas de aprendizado de máquina, que são construídos sobre informação entropia, também têm sido utilizados para classificar os dados do sensor para situações com base em características extradas dos dados do sensor.
Embora as técnicas de aprendizagem acima têm alcançado bons resultados na identificação de situação, eles precisam de uma grande quantidade de dados de treinamento para criar um modelo e estimar os parâmetros do modelo [44]. Ao treinar dados são preciosos, os pesquisadores estão motivados para aplicar técnicas de mineração de web para descobrir o conhecimento de senso comum entre as situações e objetos minerando os documentos on-line, ou seja, o que os objetos são usados ??em uma determinada atividade humana e como significativo o objeto está em identificar essa atividade [45-47]. Algumas técnicas de mineração de dados sem supervisão têm sido aplicados, bem como, incluindo-árvore de sufixos [48,49] e Jeffrey divergncia [50,51].

\subsubsection{Specification-based approaches}

Nos estgios iniciais, a pesquisa de identificao situao começa quando há alguns sensores, cujos dados são fáceis de interpretar e as relações entre os dados e situações de sensores são fáceis de estabelecer. A pesquisa consiste principalmente em abordagens baseadas em especificações que representam o conhecimento especializado nas regras da lógica e aplicar mecanismos de raciocínio para inferir situações apropriadas da entrada do sensor de corrente. Essas abordagens têm desenvolvido a partir de tentativas anteriores de primeira ordem lógica [21,22] para um modelo de lógica mais formal [20], que visa apoiar o raciocnio eficiente, mantendo o poder expressivo, para apoiar a anlise formal, e para manter a solidez e integridade de um sistema lógico. Com as suas capacidades de representao e raciocínio poderosos, ontologias têm sido amplamente aplicada [23-25,19]. As ontologias podem fornecer um vocabulário padrão de conceitos para representar o conhecimento do domínio, as especificações e as relações semânticas de situações definidas nas abordagens da lógica formal. Eles também podem fornecer motores de raciocínio de pleno direito  razão neles seguinte axiomas e restries especificadas nas abordagens da lógica formal.

Como mais e mais sensores são implantados em ambientes do mundo real para uma experiência de longo prazo, a incerteza de dados do sensor começa a ganhar atenção. Para lidar com a incerteza, as técnicas baseadas em lógica tradicionais precisam ser incorporados com outras técnicas probabilsticas [26]: onde a segurança associada com uma situação inferida, n é o número de condições que contribuem para a identificação dessa situação, wi  a massa de uma determinada condição, e ? (xi)  o grau em que a condição é satisfeita pelo actual dados do sensor.

A fórmula geral acima revela dois problemas na identificação situação. Em primeiro lugar, a satisfação de uma condição não é de modo nítido verdadeira ou falsa, o que deve levar em conta é a imprecisão dos dados do sensor. A lógica fuzzy, com a sua força para lidar com a imprecisção, foi aplicado para resolver esta questão [27]. Em segundo lugar, não contribui para cada condição identificar uma situao para o mesmo grau, de modo que o problema torna-se a forma de identificar o significado de cada prova, como resolver evidência de oposição, e como agregar evidência. Teorias como a teoria da evidência de Dempster-Shafer têm sido utilizados para resolver este problema [28,29].

%}


%------------------------------------------------------------------------------

\section{Projetos em Conscincia de Situao Explorando Regras}

Nesta seção serão descritos os principais trabalhos que se utilizam de regras para prover consciência de situação.

\subsection{CoCA - Hybrid Approach to Collaborative Context-Aware Service Platform for Pervasive Computing}

CoCA trabalho com...

\subsection{WComp, a Middleware for Ubiquitous Computing}

WComp é um abordagem...

\subsection{SCENE - A Rule-Based Platform for Situation Management}

SCENE trata de...

\subsection{CADEL - Framework and Rule-based Language for Facilitating Context-aware Computing using Information Appliances}

CADEL é...

%------------------------------------------------------------------------------

\section{Consideraes Finais}

Apresentar as consideraes finais do captulo, apresentar as consideraes finais do captulo, apresentar as consideraes finais, apresentar as consideraes finais do captulo, apresentar as consideraes finais do captulo, apresentar as consideraes finais, apresentar as consideraes finais do captulo, apresentar as consideraes finais do captulo, apresentar as consideraes finais do captulo.
Apresentar as consideraes finais do captulo, apresentar as consideraes finais do captulo, apresentar as consideraes finais, apresentar as consideraes finais do captulo, apresentar as consideraes finais do captulo, apresentar as consideraes finais, apresentar as consideraes finais do captulo, apresentar as consideraes finais do captulo, apresentar as consideraes finais do captulo.
Apresentar as consideraes finais do captulo, apresentar as consideraes finais do captulo, apresentar as consideraes finais, apresentar as consideraes finais do captulo, apresentar as consideraes finais do captulo, apresentar as consideraes finais, apresentar as consideraes finais do captulo, apresentar as consideraes finais do captulo, apresentar as consideraes finais do captulo.
Apresentar as consideraes finais do captulo, apresentar as consideraes finais do captulo, apresentar as consideraes finais, apresentar as consideraes finais do captulo, apresentar as consideraes finais do captulo, apresentar as consideraes finais, apresentar as consideraes finais do captulo, apresentar as consideraes finais do captulo, apresentar as consideraes finais do captulo.

%------------------------------------------------------------------------------

\bibliographystyle{unsrt}
\bibliography{BIBFile}

%------------------------------------------------------------------------------
%Biografia dos autores

\begin{IEEEbiography}[{\includegraphics[width=1in, height=1.25in, clip, keepaspectratio]{imagens/fotodoautor}}]{Caue Duarte}
Possui graduação em Análise e Desenvolvimento de Sistemas pela Universidade Norte do Paraná. Técnico em Eletrônica pelo Instituto Federal Sul-Rio-grandense. Possui Pós-Graduação com Enfase em Educação à Distância , atualmente é mestrando no Programa de Pós-Graduação em Computação da UFPel (Mestrado em Ciência da Computação). Atua como técnico em Sistemas de Informação da Universidade Federal de Pelotas, onde desenvolve sistemas, interfaces e realiza modelagem de sistemas/banco de dados e estudo de casos e é professor do Instituto Educacional Dimensão, no curso de Análise de Sistemas. 
%Quanto a foto, no deve ser de corpo inteiro (deve ser similar ao avatar de exemplo) e convertida em escala de cinza.
\end{IEEEbiography}

\begin{IEEEbiography}[{\includegraphics[width=1in, height=1.25in, clip, keepaspectratio]{imagens/fotodoautor.png}}]{Alexandre Gomes da Costa}
Aluno Especial do Programa de Ps-Graduao em Computao da Universidade Federal de Pelotas. Possui Bacharelado (2008) em Cincia da Computao pela Universidade Federal de Pelotas.
%Quanto a foto, no deve ser de corpo inteiro (deve ser similar ao avatar de exemplo) e convertida em escala de cinza.
\end{IEEEbiography}

\end{document}


