%%%%%%%%%%%%%%%%%%%%%%%%%%%%%%%%%%%%%%%%%
% Simple Sectioned Essay Template
% LaTeX Template
%
% This template has been downloaded from:
% http://www.latextemplates.com
%
% Note:
% The \lipsum[#] commands throughout this template generate dummy text
% to fill the template out. These commands should all be removed when 
% writing essay content.
%
%%%%%%%%%%%%%%%%%%%%%%%%%%%%%%%%%%%%%%%%%

%----------------------------------------------------------------------------------------
%	PACKAGES AND OTHER DOCUMENT CONFIGURATIONS
%----------------------------------------------------------------------------------------

\documentclass[12pt]{article} % Default font size is 12pt, it can be changed here

\usepackage[brazil]{babel}
\usepackage[utf8]{inputenc}
\usepackage{geometry} % Required to change the page size to A4
\geometry{a4paper} % Set the page size to be A4 as opposed to the default US Letter

\usepackage{graphicx} % Required for including pictures

\usepackage{float} % Allows putting an [H] in \begin{figure} to specify the exact location of the figure
\usepackage{wrapfig} % Allows in-line images such as the example fish picture

\usepackage{lipsum} % Used for inserting dummy 'Lorem ipsum' text into the template

\linespread{1.2} % Line spacing

%\setlength\parindent{0pt} % Uncomment to remove all indentation from paragraphs

\graphicspath{{Pictures/}} % Specifies the directory where pictures are stored

\begin{document}

%----------------------------------------------------------------------------------------
%	TITLE PAGE
%----------------------------------------------------------------------------------------

\begin{titlepage}

\newcommand{\HRule}{\rule{\linewidth}{0.5mm}} % Defines a new command for the horizontal lines, change thickness here

\center % Center everything on the page

\textsc{\LARGE Universidade Federal de Pelotas}\\[1.5cm] % Name of your university/college
%\textsc{\Large Keeping a Beat on the Heart}\\[0.5cm] % Major heading such as course name
%\textsc{\large Arrhythmia Monitoring System (AMS)}\\[0.5cm] % Minor heading such as course title

\HRule \\[0.4cm]
{ \huge \bfseries Survey of Context Provisioning Middleware}\\[0.4cm] % Title of your document
\HRule \\[1.5cm]

\begin{minipage}{0.4\textwidth}
\begin{flushleft} \large
\emph{Author:}\\
Alexandre Costa \\
Caue Duarte
\end{flushleft}
\end{minipage}

\vfill % Fill the rest of the page with whitespace

\end{titlepage}

%----------------------------------------------------------------------------------------
%	TABLE OF CONTENTS
%----------------------------------------------------------------------------------------

%\tableofcontents % Include a table of contents

%\newpage % Begins the essay on a new page instead of on the same page as the table of contents 

%----------------------------------------------------------------------------------------
%	PERGUNTAS
%----------------------------------------------------------------------------------------

\section{Perguntas} % Major section

\begin{itemize}
	\item O texto fala em coleta de dados brutos sobre os usuários e seu ambiente aplicando diferentes técnicas de raciocínio, quais são essas técnicas? 
	\item Como são utilizadas essas técnicas?
	\item Sistemas sensíveis ao contexto são normalmente concebidos com a adoção de middleware projetados em camadas, como se da a distribuição e funcionamento destas camadas?
	\item Quando um grupo de sensores passam a formar um contexto?
	\item No subitem B (Context-aware Systems) da introdução fala que cada camada baseia-se na informação disponibilizada pela camada abaixo dela, em seguida é citado um exemplo que a Context Processing Layer usa dados coletados da Data Acquisition Layer. Qual ou quais camadas representam a Data Acquisition Layer?
	\item No mesmo paragrafo fala também que Applications Layer interage com a Context Processing Layer que não tem ligação direta com a Applications Layer e no inicio do paragrafo fala que cada camada baseia-se na informação disponibilizada pela camada abaixo dela. Porque Applications Layer interage com a Context Processing Layer se ela não esta logo abaixo?
\end{itemize}


\end{document}

