%-------------------
% Arquivo: atividade2.tex
% Autor: Alexandre Gomes da Costa
%-------------------

\documentclass[a4paper,12pt]{article}

% Pacotes
\usepackage[brazil]{babel}
\usepackage[utf8]{inputenc}
\usepackage[hmargin=2cm,vmargin=3.5cm,bmargin=2cm]{geometry}

% Fonte Arial
\renewcommand{\rmdefault}{phv}
\renewcommand{\sfdefault}{phv}

% 
\title{\Large{\textbf{Arrhythmia Monitoring System (AMS)}}}
\author{\textbf{Alexandre Costa}\\
\normalsize{Universidade Fedela de Pelotas | Programa de Pós-Graduação em Computação (PPGC)}}
\date{\null}

\begin{document}
	\maketitle
	\thispagestyle{empty}
	\normalsize{
		Soluções avançadas de monitoramento utilizando tecnologias telecomunicação são utilizados para o diagnóstico remoto ECG, e The American College of Cardiology (ACC) e da American Heart Association (AHA) publicaram orientações para a eletrocardiografia ambulatorial. O uso das telecomunicações para o diagnóstico remoto está crescendo rapidamente, e existem vários produtos e projetos dentro gravação de ECG móvel utilizando soluções de Internet, tecnologia Bluetooth, telefones celulares, implementações baseadas em WAP e as redes locais sem fio, WLAN. Um sistema de diagnóstico remoto integrando telemetria digital foi desenvolvido, com um módulo de paciente sem fios, uma estação de assistência domiciliária e um posto clínico remoto. Tradicionalmente, 24/72 h sistemas de ECG de gravação como "-Holter" pode hoje usar construído em telefones celulares para enviar informações para o hospital, mas é usado principalmente com uma unidade de gravação que tem fisicamente para ser transportado para o hospital para análises.
	}

\end{document}
