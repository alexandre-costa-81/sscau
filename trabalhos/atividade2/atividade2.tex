%-------------------
% Arquivo: atividade2.tex
% Autor: Alexandre Gomes da Costa
%-------------------

\documentclass[a4paper,12pt]{article}

% Pacotes
\usepackage[brazil]{babel}
\usepackage[utf8]{inputenc}
\usepackage[hmargin=2cm,vmargin=3.5cm,bmargin=2cm]{geometry}

% Fonte Arial
\renewcommand{\rmdefault}{phv}
\renewcommand{\sfdefault}{phv}

% Opening
\title{\Large{\textbf{Resumo do artigo: "Visão geral de pesquisas baseadas em computação ubíqua de publicações ciêntificas"}}}
\author{\textbf{Alexandre Costa}\\
\normalsize{Universidade Fedela de Pelotas | Programa de Pós-Graduação em Computação (PPGC)}}
\date{\null}

\begin{document}
	\maketitle
	\thispagestyle{empty}
	\normalsize{
		Este artigo traz uma visão geral e análise sobre pesquisas relacionadas à computação ubíqua nacionais e internacionais. A pesquisa foi dividida em duas partes. Primeiro, foram coletadas informações sobre todos os artigos publicados nas tres maiores conferências de computação ubíqua e foi realizado um processo de mineração de dados extraindo estatísticas como: autores mais produtivos e instituições. Depois, foram analisados todos os artigos publicados em 2010 e 2011 no grupo TOPint, criando uma taxonomia de pesquisa de computação ubíqua recente. O TOPint é um grupo que reúne conferencias na area de computação ubíqua onde os tres principais são Ubicomp, pervasive e percom analisados neste trabalho. Também foi analisado o SBCUP.\\
		Na analise das publicações foi posível observar que foram publicados 397 artigos em treze edições da Ubicomp, 308 artigos em nove edições do Percomp e 203 trabalhos em nove edições do Pervasive. No total de 908 artigos e 2239 autores distintos.\\
		Com relação a taxonomia os trabalhos foram dividias em 7 categorias principais context-aware, privacy / security, innovative technology, invisible computing, HCI evaluation, software design e resource limitation. Como um artigo podia estar relacionado a mais de uma categoria foi levado em concideração a principal contribuição do artigo. O artigo identificou e descreveu cada categoria.\\
		O artigo também cassificou os trabalhos apresentados no SBCUP entre 2010 e 2011 nas taxonomias propostas anteriormente. No SBCUP a categoria mais frequente foi software design com 52 por cento dos trabalhos. Já os trabalhos internacionais tiveram context-aware como categoria com maior numeo de artigos relacionados com 35 por cento.\\
		Este artigo apresentou uma analise de artigos publicados no grupo TOPint e foi proposto uma taxonomia de pesquisas em computação ubíqua. Foi proposto também a manutenção dessa taxonomia para trabalhos futuros.
	}

\end{document}

