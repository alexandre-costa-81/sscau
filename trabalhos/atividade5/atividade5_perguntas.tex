%%%%%%%%%%%%%%%%%%%%%%%%%%%%%%%%%%%%%%%%%
% Simple Sectioned Essay Template
% LaTeX Template
%
% This template has been downloaded from:
% http://www.latextemplates.com
%
% Note:
% The \lipsum[#] commands throughout this template generate dummy text
% to fill the template out. These commands should all be removed when 
% writing essay content.
%
%%%%%%%%%%%%%%%%%%%%%%%%%%%%%%%%%%%%%%%%%

%----------------------------------------------------------------------------------------
%	PACKAGES AND OTHER DOCUMENT CONFIGURATIONS
%----------------------------------------------------------------------------------------

\documentclass[12pt]{article} % Default font size is 12pt, it can be changed here

\usepackage[brazil]{babel}
\usepackage[utf8]{inputenc}
\usepackage{geometry} % Required to change the page size to A4
\geometry{a4paper} % Set the page size to be A4 as opposed to the default US Letter

\usepackage{graphicx} % Required for including pictures

\usepackage{float} % Allows putting an [H] in \begin{figure} to specify the exact location of the figure
\usepackage{wrapfig} % Allows in-line images such as the example fish picture

\usepackage{lipsum} % Used for inserting dummy 'Lorem ipsum' text into the template

\linespread{1.2} % Line spacing

%\setlength\parindent{0pt} % Uncomment to remove all indentation from paragraphs

\graphicspath{{Pictures/}} % Specifies the directory where pictures are stored

\begin{document}

%----------------------------------------------------------------------------------------
%	TITLE PAGE
%----------------------------------------------------------------------------------------

\begin{titlepage}

\newcommand{\HRule}{\rule{\linewidth}{0.5mm}} % Defines a new command for the horizontal lines, change thickness here

\center % Center everything on the page

\textsc{\LARGE Universidade Federal de Pelotas}\\[1.5cm] % Name of your university/college

\HRule \\[0.4cm]
{ \huge \bfseries Capítulo 2 - Arquiteturas, do Livro “Tanenbaum & van Steen. Sistemas Distribuídos: Princípios e Paradigmas. 2. ed., Prentice-Hall, 2007.”}\\[0.4cm] % Title of your document
\HRule \\[1.5cm]

\begin{minipage}{0.4\textwidth}
\begin{flushleft} \large
\emph{Author:}\\
Alexandre Costa\\
Caue Duarte
\end{flushleft}
\end{minipage}

\vfill % Fill the rest of the page with whitespace

\end{titlepage}

%----------------------------------------------------------------------------------------
%	TABLE OF CONTENTS
%----------------------------------------------------------------------------------------

%\tableofcontents % Include a table of contents

%\newpage % Begins the essay on a new page instead of on the same page as the table of contents 

%----------------------------------------------------------------------------------------
%	ABISTRACT
%----------------------------------------------------------------------------------------

\section{Perguntas} % Major section

\begin{enumerate}
	\item Qual a implicação de utilizar, na arquitetura cliente/servidor protocolos sem conexão?
%R. Maior velocidade, porém mais suscetível a falhas.
	\item Qual motivo para na arquitetura cliente/servidor não haver distinção clara  entre cliente e servidor ?
%R. Pois , por exemplo, quando utilizados em sistemas distribuídos, servidores também fazem requisições para outros servidores e até mesmo clientes. 
	\item Diferencie as camadas do estilo arquitetural em camadas e cite um exemplo de aplicação\\
a. Nível de interface com o usuário\\
b. Nível de processamento.\\
c. Dados
%R. Buscador,  pois normalmente é baseadona arquitetura cliente/servidor , possuindo uma camada de visualização, servidores com regras de negócios com algoritmos inteligentes (adwords, propagandas) e finalmente uma camada de dados  com imensos volumes de dados.
	\item Bancos relacionais são a soluções ideais para toda a gama de problemas da computação ?
%R. Não pois apenas se restringem ao domínio dos dados que podem ser representados no modelo entidade-relacionamento.  Exemplo de caso em que não são a solução ideal é para armazenar figuras (CAD por exemplo) onde modelos orientados a objetos são mais eficazes.
	\item Diferencie fat clients e thin clients.
%R. Diferenciam-se pela quantidades de camadas funcionando no cliente. Quanto mais enxuto, mais próximo somente da camada de visualização executa no cliente, mais thin o cliente. Quanto mais pesado, mais próxima da camada de dados executa no cliente, mais fat o cliente.
	\item Cite um exemplo de arquitetura de três diviões, em termos físicos.
%R. Um exemplo onde muitas vezes vemos uma arquitetura de três divisões, é a organização de sites Web. Neste caso, um servidor Web age como ponto de entrada para um site, passando requisições para um servidor de aplicação no qual ocorre o processamento propriamente dito. Por sua vez, esse servidor de aplicação interage com um servidor de banco de dados.

\end{enumerate}

\end{document}

